% Introduction: Explain the motivation behind analyzing this dataset.

In the era of data-driven decision-making, analyzing customer behavior has become paramount for businesses aiming to fine-tune their marketing strategies and enhance customer satisfaction. This study delves into a rich dataset encompassing various attributes related to customers' demographic profiles such as, \textit{Year\_Birth}, \textit{Education}, \textit{Marital\_Status}. We aim the exploration of how marital status (\textit{Marital\_Status}) correlates with annual household income, which in turn influences spending patterns across various product categories such as wines, fruits, and other goods over a biennial span. This inquiry not only aims at mapping out correlations but also endeavors to model predictive insights that could guide stakeholders in crafting more personalized, data-informed approaches towards engaging diverse customer segments. 

A model that can classify the income level of each user from marketing dataset could be useful for various stakeholders, including:
\begin{enumerate}
    \item Marketing and advertising companies:
    \begin{enumerate}
        \item These companies could use such a model to segment their target audience more effectively based on predicted income levels.
        \item They could tailor their marketing campaigns, messaging, and advertising content to resonate better with different income groups, increasing the likelihood of successful conversions and sales.
    \end{enumerate}
    
  
    \item E-commerce and retail businesses:
    \begin{enumerate}
        
    \item By predicting customers' income levels, these businesses could personalize product recommendations, promotions, and user experiences accordingly.
    \item They could offer more relevant products, services, or content to different customer segments, improving customer satisfaction and potentially increasing sales.
    \end{enumerate}
    
\end{enumerate}

We will also use Ensemble Learning and Model Averaging in analyzing the data. By training diverse models, such as k-nearest neighbors, artificial neural networks, decision trees, random forests, naive Bayes classifiers, and support vector machines (SVMs), we can combine their predictions through ensemble learning techniques like bagging, boosting, or stacking. Ensemble methods often outperform individual models by leveraging the strengths of each model and compensating for their weaknesses, leading to improved accuracy and robustness.
